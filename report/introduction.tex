\section{Introduction}
Defining good threat models can be very helpful in improving existing systems, or contribute in building of better, safer and more fault tolerant systems. Threat models are especially important in Software engineering and Computer Science in general because it is a relatively new field of research. A threat model can help us to understand how a particular system reacts to a situation that is different from normal operating conditions. The limits that a threat model imposes makes it easier to discuss and reach conclusions in a predefined scenario.

In this paper, the threat models are picked up in that way so they can be used to compare the ability of the centralised and decentralised computer systems to deal with circumstances other than normal. Because centralised and decentralised systems have much broader meaning, the focus is narrowed down to centralised computer networks and decentralised (distributed) computer networks and applications that are dependent on these particular network architecture. 

The most popular applications among the users now are the social networking sites, where the absolute giant among them is Facebook, which has recently passed one billion users \cite{web:facebookpassesbillion}. Other popular social networks are Twitter, Google Plus, MySpace, LinkedIn. Conclusions for decentralised computer systems will be made after observing how threat models affect peer-to-peer networks. The whole set of decentralised networks has been reduced to just peer-to-peer networks, because they have the essential characteristics that are of interest. Also there is whole class of applications between centralised and decentralised model like e-mail, DNS and XMPP that have both centralised and decentralised characteristics. They are not subject of interest in this paper.

According to Internet World Statistics research \cite{web:internetworldstats} the Internet penetration percent in the most progressive countries can go even above 60\%. Examples are Northern America (78.6\%), Australia and Oceania (67.5\%) and Europe (61.3\%). Roughly every third person on the planet has access to the Internet, and has power to share information. 

The figures of Internet penetration \cite{web:internetworldstats} in human societies, implies that people need this tools even in the most harsh conditions. Especially important are problems which can occur in the infrastructure of the computer systems or threats related to security, with special emphasis on security of the users data. Threat models have been modelled to target both of the mentioned aspects of potential flaws. The most important thing is that this paper does not capture the whole picture of a threat, but rather than that certain characteristics had been taken to model the newly occurred situation.

The rest of the paper is structured as follows. In the Section 2 the methods for research to produce this paper are presented. Every key phrase used in search for relevant references has been given along with the used search engines. Methods for building the models are also presented in this section. Next, in the Section 3 there is a short summary of all three threat models, with a depiction of the most important features (conditions). In the Sections 4, 5 and 6 respectively, the previously defined threat models had been applied on both types of network architectures. By giving a detailed description of the circumstances, the relation with a particular threat model can be seen. Threat models have been applied to both centralised and decentralised systems under the same circumstances. The predicted reactions of the systems under the new conditions are used in the Section 7 where the conclusion is presented based on the previous findings.
