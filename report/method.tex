\section{Method}
The literature study conducted focused mainly on specific search terms. The search terms used were "Adversary Model", "Adversary Models", "Distributed knowledge", "Peer to peer security", "Censorship in centralized networks", "Censorhip in P2P networks", "Threat Model" and "Threat Models".
These initial terms were chosen by the authors and used in the "Discovery" search engine operated by "M\"{a}lardalen University" and powered by EBSCOhost.
The EBSCOhost search engines goes through several different journals such as IEEE and ACM among others.
As alternative search engine Google scholar was used.
During the literature study other sources were found by following references and following suggestions by others in this research area.

After the initial literature study three different threat models were chosen that was found interesting by the authors to study in more detail.
As selection criteria for the threat models two main criteriums were of most importance. 
The first one was that the threat models must be neutral in terms of centralised or decentralised systems as this is the comparison that is made.
The second one was that the selected threat models should be current in the sense that the threat could be realised in today's society.
To make it clear the chosen threat models have been composed by the authors with inspiration from different adversary and threat models found in different papers.

The last part in the research before writing this paper was to decide upon the structure of the paper. The chosen structure is by separation of threat models as it is important not to mix different scenarios.
