\section{Threat models}
This paper will focus on different threat models and how a centralised architecture and a distributed architecture (peer-to-peer) will deal with each threat.
In this chapter all threat models will be listed with references to events that has occurred in the past that makes each threat real.

\subsection{Natural disaster}
This is the first threat model which will focus on infrastructure failure.

A natural disaster occurs that wipes out the infrastructure that powers the cell phone towers as well as other communication systems that under normal circumstances are available such as DSL, fiber or landlines.
\begin{itemize}
  \item The only devices left operating are the ones with battery such as laptops and cell phones.
  \item These devices have one or more features that allows them to connect to other similar devices, example of this could be WiFi, Bluetooth, IR or USB drive.
\end{itemize}

\subsection{Privacy}
This is the second threat model which will focus on privacy concerns in social networks.

Social media such as Facebook, Twitter and Google Plus becomes bigger and bigger for each day that passes by.
It's not easy to stay away from social media today as the services have become a good part of everyday life.
If you're not in, you are an outsider. 
With the increasing amount of private data about all of the services' respective users what would happened if that information was compromised.

\begin{itemize}
  \item Users share private information with friends and families and believes that this information will be restricted to only the people they allow read access to.
  \item The private information is someday accessed by someone who should not have access to it. 
\end{itemize}

\subsection{Censorship}
The last threat model presented in the paper will focus on the problem of data censorship present on the Internet.

As the number of users grows every day, the amount of data present into the Internet infrastructure grows with even bigger rate.
Users tend to share different type of information, without taking much care about the contents of it.
There are some types of information like pornography, piracy, hate-speech  that should be carefully monitored or in some severe cases even banned.
But sometimes this becomes excuse for the authorities to sniff someones data and to decide what is right and what is wrong.

\begin{itemize}
  \item Users want to share some information outside their physical reach.
  \item Users data has been compromised by some third parities (authorities) that do not want that information do be spread.
  \item Banned global services in purpose of propaganda.
\end{itemize}
