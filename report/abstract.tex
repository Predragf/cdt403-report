\begin{abstract}
Through time computer networks have evolved into fault tolerant systems, that have substantial level of reliability. Still they are far from perfect, and we can find numerous potential threats that can affect their proper functioning. 

Constructing a high quality threat models and studying their effect on different types of networks can be a very complex process that requires a substantial amount of time and financial founds. Instead of that, in this paper we tend to desing rather simple threat models with high probbability of occurance, and by applying them to a specific network architecture or applications we try to analyze their effect on centralized and decentralized computer networks.

The main focus of this paper are centralized and decentralized network architectures. In order to test their resistance to threats, we have introduced three threat models, which are designed in that fashion to capture specific aspects of the particular architecture. We put our emphasis on how centralized and decentralized computer network deal with infrastructure and data integrity threats. By applying our threat models, we can see the potential flows into the architecture, which will be used to summarize which network architecture is more tolerant to that specific threat. 
As a conclusion we will summarize how our threat models changed the characteristics of the centralized and decentralized networks and propose best solution in situations with high probability of particular threat model.
\end{abstract}
