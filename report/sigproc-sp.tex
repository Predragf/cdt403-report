% THIS IS SIGPROC-SP.TEX - VERSION 3.1
% WORKS WITH V3.2SP OF ACM_PROC_ARTICLE-SP.CLS
% APRIL 2009
%
% It is an example file showing how to use the 'acm_proc_article-sp.cls' V3.2SP
% LaTeX2e document class file for Conference Proceedings submissions.
% ----------------------------------------------------------------------------------------------------------------
% This .tex file (and associated .cls V3.2SP) *DOES NOT* produce:
%       1) The Permission Statement
%       2) The Conference (location) Info information
%       3) The Copyright Line with ACM data
%       4) Page numbering
% ---------------------------------------------------------------------------------------------------------------
% It is an example which *does* use the .bib file (from which the .bbl file
% is produced).
% REMEMBER HOWEVER: After having produced the .bbl file,
% and prior to final submission,
% you need to 'insert'  your .bbl file into your source .tex file so as to provide
% ONE 'self-contained' source file.
%
% Questions regarding SIGS should be sent to
% Adrienne Griscti ---> griscti@acm.org
%
% Questions/suggestions regarding the guidelines, .tex and .cls files, etc. to
% Gerald Murray ---> murray@hq.acm.org
%
% For tracking purposes - this is V3.1SP - APRIL 2009

\documentclass{acm_proc_article-sp}

\begin{document}

\title{Comparison between centralized and distributed systems and how they cope with different threat models\titlenote{Title is work in progress.}}
%\subtitle{[Extended Abstract]
%\titlenote{A full version of this paper is available as
%\textit{Author's Guide to Preparing ACM SIG Proceedings Using
%\LaTeX$2_\epsilon$\ and BibTeX} at
%\texttt{www.acm.org/eaddress.htm}}}
%
% You need the command \numberofauthors to handle the 'placement
% and alignment' of the authors beneath the title.
%
% For aesthetic reasons, we recommend 'three authors at a time'
% i.e. three 'name/affiliation blocks' be placed beneath the title.
%
% NOTE: You are NOT restricted in how many 'rows' of
% "name/affiliations" may appear. We just ask that you restrict
% the number of 'columns' to three.
%
% Because of the available 'opening page real-estate'
% we ask you to refrain from putting more than six authors
% (two rows with three columns) beneath the article title.
% More than six makes the first-page appear very cluttered indeed.
%
% Use the \alignauthor commands to handle the names
% and affiliations for an 'aesthetic maximum' of six authors.
% Add names, affiliations, addresses for
% the seventh etc. author(s) as the argument for the
% \additionalauthors command.
% These 'additional authors' will be output/set for you
% without further effort on your part as the last section in
% the body of your article BEFORE References or any Appendices.

\numberofauthors{2} 
% I've updated the number of authers ~ Christoffer 2012-10-14

%  in this sample file, there are a *total*
% of EIGHT authors. SIX appear on the 'first-page' (for formatting
% reasons) and the remaining two appear in the \additionalauthors section.
%
\author{
% You can go ahead and credit any number of authors here,
% e.g. one 'row of three' or two rows (consisting of one row of three
% and a second row of one, two or three).
%
% The command \alignauthor (no curly braces needed) should
% precede each author name, affiliation/snail-mail address and
% e-mail address. Additionally, tag each line of
% affiliation/address with \affaddr, and tag the
% e-mail address with \email.
%
% 1st. author
\alignauthor
Predrag Filipovik\\
       \email{predragku@gmail.com}
% 2nd. author
\alignauthor
Christoffer Holmstedt\\
       \email{christoffer.holmstedt@gmail.com}
}
% There's nothing stopping you putting the seventh, eighth, etc.
% author on the opening page (as the 'third row') but we ask,
% for aesthetic reasons that you place these 'additional authors'
% in the \additional authors block, viz.
% \additionalauthors{Additional authors: John Smith (The Th{\o}rv{\"a}ld Group,
%email: {\texttt{jsmith@affiliation.org}}) and Julius P.~Kumquat
%(The Kumquat Consortium, email: {\texttt{jpkumquat@consortium.net}}).}
%\date{30 July 1999}
% Just remember to make sure that the TOTAL number of authors
% is the number that will appear on the first page PLUS the
% number that will appear in the \additionalauthors section.

\maketitle
\begin{abstract}
This is our abstract.
\end{abstract}

% A category with the (minimum) three required fields
\category{H.4}{CHANGE THIS Information Systems Applications}{Miscellaneous}
%A category including the fourth, optional field follows...
\category{D.2.8}{CHANGE THIS Software Engineering}{Metrics}[complexity measures, performance measures]

\terms{Theory}

\keywords{ACM proceedings, \LaTeX, text tagging} % NOT required for Proceedings

\section{Introduction}
Defining good threat models can be very helpful in improving existing systems, or contribute in building of better, safer and more fault tolerant systems. Threat models are especially important in Software engineering and Computer Science in general because it is a relatively new field of research and there are not much threat models defined. A threat model can help us to understand how a particular system reacts to a situation that is different from normal operating conditions. The limits that a threat model imposes makes it easier to discuss and reach conclusions in a predefined scenario.

In this paper, the threat models are picked up in that way so they can be used to compare the ability of the centralised and decentralised computer systems to deal with circumstances other than normal. Because centralised and decentralised systems have much broader meaning, the focus is narrowed down to centralised computer networks and decentralised (distributed) computer networks and applications that are dependent on these particular network architecture. 

To study the characteristics of the centralised computer systems, a well known centralised web applications will be used. The most popular applications among the users now are the social networking sites, where the absolute giant between them is Facebook, which has recently passed one billion users \cite{web:facebookpassesbillion}. Other popular social networks are Twitter, Google Plus, MySpace, LinkedIn. Conclusions for decentralised computer systems will be made after observing how threat models affect peer-to-peer networks. The whole set of decentralized networks has been reduced to just peer-to-peer networks, because they have the essential characteristics that are of interest. Also there is whole class of application in between centralised and decentralised model like e-mail, DNS and XMPP that have both centralised and decentralised characteristics. They are not subject of interest in this paper.

According to Internet World Statistics research \cite{web:internetworldstats} the Internet penetration percent in the most progressive countries can go even above 60\%. Examples are Northern America (78.6\%), Australia and Oceania (67.5\%) and Europe (61.3\%). Roughly every third person on the planet has access to the Internet, and has power to share information. 

The figures of Internet penetration \cite{web:internetworldstats} in human societies, implies that people need this tools even in the most harsh conditions. Especially important are problems which can occur in the infrastructure of the computer systems or threats related to security, with special emphasis on security of the users data. Threat models have been modelled to target both of the mentioned aspects of potential flaws. The most important thing is that this paper does not captures the whole picture of a threat, but rather than that certain characteristics had been taken to model the newly occurred situation.

The rest of the paper is structured as follows. In the Section 2 the methods for research to produce this paper are presented. Every key phrase used in search for relevant references has been given along with the used search engines. Methods for building the models are also presented in this section. Next, in the Section 3 there is a short summary of all three threat models, with a depiction of the most important features (conditions). In the Sections 4, 5 and 6 respectively, the previously defined threat models has been applied on both types of network architecture. By giving a detailed description of the circumstances, the relation with a particular threat model can be seen. Threat models have been applied to both centralized and decentralized systems under the same circumstances. The predicted reaction of the systems under the new conditions is used in the Section 7 where the conclusion is presented based on the previous findings.



\section{Method}
The literature study conducted in the beginning focused mainly on specific search terms. The search terms used were "Adversary Model", "Adversary Models", "Distributed knowledge", "Peer to peer security", "Threat Model" and "Threat Models".
These initial terms were chosen by the authors and used in the "Discovery" search engine operated by "Mälardalen University" and powered by EBSCOhost.
The EBSCOhost search engines goes through several different journals such as IEEE and ACM among others.
As alternative search engine Google scholar was used.
During the literature study other sources were found by following references and following suggestions by others in this research area.

After the initial literature study three different threat models were chosen that was found interesting by the authors to study in more detail.
As selection criteria for the threat models two main criterias were of most importance. 
The first one was that the threat models must be neutral in terms of centralized or decentralized systems as this is the comparison that is made.
The second one was that the selected threat models should be current in the sense that the threat could be realised in todays society.
To make it clear the chosen threat models have been composed by the authors with inspiration from different adversary and threat models found in different papers.

The last part in the research before writing this paper was to decide upon the structure of the paper. The chosen structure is by seperation of threat models as it's important not to mix different scenarios.


\section{Threat models}
This report will focus on different threat models and how a centralised architecture and a distributed architecture (peer-to-peer) will deal with each threat.
In this chapter all threat models will be listed with references to events that has occurred in the past that makes each threat real.

\subsection{A natural disaster}
This is the first threat model which will focus on infrastructure failure.

A natural disaster occurs that wipes out the infrastructure that powers the cell phone towers as well as other communication systems that under normal circumstances are available such as DSL, fiber or landlines.
\begin{itemize}
  \item The only devices left operating are the ones with battery such as laptops and cell phones.
  \item These devices have one or more features that allows them to connect to other similar devices, example of this could be WiFi, Bluetooth, IR or USB drive.
\end{itemize}

\subsection{Private information is compromised}
This is the second threat model which will focus on privacy concerns in social networks.

Social media such as Facebook, Twitter and Google Plus becomes bigger and bigger for each day that passes by.
It's not easy to stay away from social media today as the services have become a good part of everyday life.
If you're not in, you are an outsider. 
With the increasing amount of private data about all of the services' respective users what would happened if that information was compromised.

\begin{itemize}
  \item Users share private information with friends and families and believes that this information will be restricted to only the people they allow read access to.
  \item The private information is someday accessed by someone who should not have access to it. 
\end{itemize}

\subsection{Censorship}
The last threat model presented in the paper will focus on the problem of data censorship present on the Internet.

As the number of users grows every day, the amount of data present into the Internet infrastructure grows with even bigger rate. Users tend to share different type of information, without taking much care about the contents of it. There are some types of information like pornography, piracy, hate-speech  that should be carefully monitored or in some severe cases even banned. But sometimes this becomes excuse for the authorities to sniff someones data and to decide what is right and what is wrong.

\begin{itemize}
  \item Users want to share some information outside their physical reach.
  \item Users data has been compromised by some third parities (authorities) that do not want that information do be spread.
  \item Banned global services in purpose of propaganda.
\end{itemize}


\section{How a centralised architecture design deal with specificed threat models}
Centralised with servers...sort of.


\section{How a peer-to-peer architecture design deal with specificed threat models}
Peer to peer.


\section{Conclusions}
In this paper no new information is presented.
The paper is an overview on how centralised and decentralised systems cope with different threats.

The comparisons made in each scenario show that centralised and decentralised systems have their pros and cons.
No one can say that centralised or decentralised systems will be better or worse in dealing with all types of threats.
In the threat models defined in this paper decentralised systems have a small advantage.
As an example to this in the aftermath of a natural disaster a community can on their own set up a decentralised network without having to wait for someone to fix a centralised node e.g. setting up a temporary cellular tower.

%The threats are composed in three different threat models defining three different scenarios.
It's clear that defining just one threat model and making sure that all corner cases are covered is a demanding task and a time consuming process in itself.
When writing this paper it was early on decided that the focus should be put on the comparing centralised and decentralised system instead of trying to define the perfect threat models.
For future research defining the different threat models in more detail is a first step.

%This would mean that a more organised and controlled recovery would start sooner with a decentralised network than a centralised network.

%Also in the situation were a censorship is a regularly imposed, decentralised networks have shown better results in dealing with it.
%The fact that the users are able to discover themselves without intervention of any third parity, makes the communication and data exchange between them difficult to track and monitor.
%On top of that if the data has been additionally encrypted, they can be pretty sure that their data is safe and resistant to censorship.

% As a last note it is important to remember what is trying to be achieved.
% In the threat model about privacy it is a must to define what type of privacy you discuss because it is a big difference between "institutional privacy" and "social privacy".

%\end{document}  % This is where a 'short' article might terminate

% Just comment this out if we don't need it.
%ACKNOWLEDGMENTS are optional
\section{Acknowledgments}
If we need this.


%
% The following two commands are all you need in the
% initial runs of your .tex file to
% produce the bibliography for the citations in your paper.
\bibliographystyle{abbrv}
\bibliography{sigproc}  % sigproc.bib is the name of the Bibliography in this case
% You must have a proper ".bib" file
%  and remember to run:
% latex bibtex latex latex
% to resolve all references
%
% ACM needs 'a single self-contained file'!
%
%APPENDICES are optional
%\balancecolumns
%\appendix
%Appendix A
%\section{Headings in Appendices}
%The rules about hierarchical headings discussed above for
%the body of the article are different in the appendices.
%In the \textbf{appendix} environment, the command
%\textbf{section} is used to
%indicate the start of each Appendix, with alphabetic order
%designation (i.e. the first is A, the second B, etc.) and
%%a title (if you include one).  So, if you need
%hierarchical structure
%\textit{within} an Appendix, start with \textbf{subsection} as the
%highest level. Here is an outline of the body of this
%document in Appendix-appropriate form:
%\subsection{Introduction}
%\subsection{The Body of the Paper}
%\subsubsection{Type Changes and  Special Characters}
%\subsubsection{Math Equations}
%\paragraph{Inline (In-text) Equations}
%\paragraph{Display Equations}
%\subsubsection{Citations}
%\subsubsection{Tables}
%\subsubsection{Figures}
%\subsubsection{Theorem-like Constructs}
%\subsubsection*{A Caveat for the \TeX\ Expert}
%\subsection{Conclusions}
%\subsection{Acknowledgments}
%\subsection{Additional Authors}
%This section is inserted by \LaTeX; you do not insert it.
%You just add the names and information in the
%\texttt{{\char'134}additionalauthors} command at the start
%of the document.
%\subsection{References}
%Generated by bibtex from your ~.bib file.  Run latex,
%then bibtex, then latex twice (to resolve references)
%to create the ~.bbl file.  Insert that ~.bbl file into
%the .tex source file and comment out
%the command \texttt{{\char'134}thebibliography}.
% This next section command marks the start of
% Appendix B, and does not continue the present hierarchy
%\section{More Help for the Hardy}
%The acm\_proc\_article-sp document class file itself is chock-full of succinct
%and helpful comments.  If you consider yourself a moderately
%experienced to expert user of \LaTeX, you may find reading
%it useful but please remember not to change it.
\balancecolumns
% That's all folks!
\end{document}
