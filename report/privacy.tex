\section{Privacy}
As stated in earlier chapters social media is on the rise.
Most of these networks are centralised e.g. Facebook, Twitter and Google plus.
Users share anykind of private information in the belief that it is their own information only accesible by friends that they know.
This is though far from true in many cases.
As an example social media websites search through all information submitted by users to be able to target the user with tailored ads.
This is not appreciated by most americans as a telephone survey conducted in 2009 conclude \cite{turow2009}.
It's not far fetched in this scenario to believe that similiar results would be aquired in other parts of the world.

Other studies show that "institutional" privacy e.g. the scenario above when a company analysis its users behaviour and shared information is not the main concern \cite{raynes-goldie2010}.
Instead the users in the study thought that "social" privacy is more important.
Social privacy include scenarios from friend requests (from the "wrong" person e.g. the users boss or teacher) to posting daily messages that should be suitable for all friends.
Some participants in the study was troubled by the lack of control of shared data on facebook.
As an example it's not often that the same message is percieved in the same way by your boss and your best friend.
To take the example one step further a message posted during a night out with friends in the weekend could be misinterpreted by the users boss and have consequences the next workday.

It's important to remember that what we percieve as problems with privacy in different services one day may not be a problem the next day.
In 2006 Facebook introduced the news feed function or the Facebook wall as it's called today.
With this release all information available in every users network of friends got aggregated to one simple list of actions taken during the last period of time.
The information was available all the time but not that easy to get an overview of, now the news feed became that overview and it became the main feature of Facebook.
Today, in 2012, facebook groups form for all kinds of reason and this was the case back in 2006 as well.
In a few days the group "Students Against Facebook News Feed" had over 700 thousand people. \cite{boyd2008}
This clearly shows that privacy is of main concern for many people but the news feed is still the main feature on Facebook today after some privacy settings were added.

One interesting feature about privacy is that users that know about loopholes in security will use them even if the same action against themselves is not appreciated.
Raynes-Goldie found that participants in her study used direct URLs to images on Facebook that was normally hidden behind privacy settings \cite{raynes-goldie2010}.

TODO. Write about privacy in p2p network. friendlySurveillance.
